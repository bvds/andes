\documentclass[11pt,letterpaper]{article}
\usepackage{graphics}
\usepackage{fullpage}
\addtolength{\voffset}{0.5in}

%
%  Cool LaTeX resource:
%   http://en.wikibooks.org/wiki/LaTeX

\begin{document}
\title{Bayesian Knowledge Tracing and the Identifiability Problem}
\author{Brett van de Sande}

\maketitle

\begin{abstract}
The Bayesian Knowledge tracing model is very widely used in
modeling student learning.  Unlike most Hidden Markov Models,
it is simple enough that it can be solved analytically, which
gives some insight into its behavior.  In particular, we show that
it is, in fact, a model with three parameters and has the 
functional form of an exponential.
\end{abstract}


\section{Introduction}

The Bayesian Knowledge Tracing (BKT) model was first introduced by Corbett
and Anderson~\cite{anderson}.  Since then, it has been widely applied
in studies of student learning.  In addition, it often serves as the starting
%
%   Need examples of BKT-based models, only look at examples
%   where identifiability problem exists.
%
point for more complicated models of learning, for example~\cite{bca,brunskill}.  
While using this model,  number of authors have noted have noted the 
existence of what is called the ``identifiability problem'' \cite{beckchang}.    
That is, there seem to be certain combinations of the model 
parameters that produce identical models.   

The BKT model is commonly expressed as a Hidden Markov Model (HMM).
A typical HMM must be solved numerically to find its functional form.
However, the BKT model is simple enough that it can be solved analytically,
which we will do here.  We will see that the model, in functional
form, is an exponential.  Moreover, we will see that it is, in fact,
a three parameter model, explaining the identifiability problem.

%There are several reasons why a HMM is a good represention.  First,
%the parameters of the HMM are often thought to have some physical
%significance.  In BKT, for instance, $P_j$ is the probability that the
%student knows the skill at a certain point of time:  it is a simple
%mental model of the student.  Thus, a distinction can be made between
%the student's mental state and their actual behavior (whether they
%can apply the skill correctly).   In contrast, a more behaviorist
%approach would eschew $P_j$ in favor of a model of actual behavior.


\section{Bayesian Knowledge Tracing}

The Bayesian Knowledge Tracing model~\cite{anderson} has four parameters:
%
\begin{itemize}
   \item $P_0$ is the initial probability of knowing a skill.
   \item $P(G)$ is probability of guessing correctly, if the student        
         doesn't know the skill.
   \item $P(S)$ is probability of slips, if student does know the skill.
   \item $P(L)$ is probability of learning the skill if the student 
         does not know the skill.  Note that $P(L)$ is assumed to 
         be constant over steps.
\end{itemize}
%
Let $P_j$ be the probability that the student knows the skill at 
step $j$. According to the model,  $P_j$ can
be determined in terms of the previous opportunity:
%
\begin{equation}
          P_j = P_{j-1} + P(L)\left(1-P_{j-1}\right)  \; . \label{recurse}
\end{equation}
%
According to this model, the probability that the student actually gets
opportunity $j$ correct is:
%
\begin{equation}
         P_j(C) = P(G)\left(1-P_j\right) + \left(1-P(S)\right) P_j \; . \label{pnc}
\end{equation}
%
(Unlike the four model parameters above, there isn't a consistent
notation for $P_j(C)$ in the literature.)

This model can be  solved exactly.  First, we rewrite (\ref{recurse}) in a
more suggestive form:
%
\begin{equation}
        1-P_j = \left(1-P(L)\right) \left(1-P_{j-1}\right) \; .
\end{equation}
%
One can show that this recursion relation has solutions of the form:
%
\begin{equation}
            1-P_j = \left(1-P(L)\right)^j\left(1-P_0\right) \; .
	    \label{sol}
\end{equation}
%
%
Substituting (\ref{sol}) into (\ref{pnc}), we get:
%
\begin{equation}
         P_j(C) = 1-P(S) -\left(1-P(S)-P(G)\right) \left(1-P_0\right)
                   \left(1-P(L)\right)^j \; . \label{pncsoln}
\end{equation}
%
Note that the functional form of $P_j(C)$ is a function of {\em three}
parameters:  $P(S)$, $P(L)$, and $\left(1-P(S)-P(G)\right) \left(1-P_0\right)$.
This degeneracy of the model was first noticed by Beck and 
Chang~\cite{beckchang} where they call it the ``identifiability
problem.''   In their paper, they note that multiple
combinations of $P(G)$ and $P_0$ give exactly the same $P_j(C)$, but
fail to explain why this is the case.

The functional form of (\ref{pncsoln}) is an exponential.
If we define 
$A=\left(1-P(S)-P(G)\right) \left(1-P_0\right)$ and
$\beta=-\log(1-P(L))$, then we can rewrite (\ref{pncsoln}) in 
a clearer form:
%
\begin{equation}
         P_j(C) = 1-P(S) -A e^{-\beta j} \, ;
\end{equation}
%
see Fig.~\ref{bktgraph}.

\begin{figure}
\centering\includegraphics{exponential.eps}
\caption{The Bayesian Knowledge Tracing model in functional form. 
          $P_j(C)$ the probability of  the student getting step $j$ correct.}
 \label{bktgraph}
\end{figure}

\section{Conclusion}

In conclusion, the Bayesian Knowledge Tracing model, when expressed
in functional form, is an exponential function with three parameters.
It is unclear how the identifiability problem has affected previous
results.  Certainly, any study that relies on a value for $P(G)$ or
$P_0$ as output from a fit to student data would be affected.
On the other hand, studies that rely only on $P(S)$ or $P(L)$ from
a fit to student data should be OK.

Some authors have addressed the identifiability problem by 
fixing $P_0$ using some additional constraint~\cite{fix}.  
This is only necessary in the case
where, after fitting the model to data, $P(G)$ a being used as an output.

% This needs work!  Should find one or more examples and investigate!
In the case of more complicated models, such as seen 
in~\cite{bca,brunskill}, analytic solutions are not available.  However,
there are qualitative methods that can be applied~\cite{qualitative},
so the solutions can be classified and asymptotic values can 
be found.

% Question:  can Markov chains produce power law behavior? YES!
% Need supporting literature for power law vs. exponential.

Finally, we see that the functional form of BKT corresponds 
to an exponential, rather than a power law.  
Heathcote, Brown, and Mewhort~\cite{powerlaw}
argue that learning for individuals is better described by
exponentials while (as shown in earlier studies) learning averaged
over individuals is better described by a power law function.
This would suggest that BKT may be more appropriate for describing
individual learners.


%Equations (1) and (2) of \cite{brunskill} 
%are equivalent to the first three equations in \cite{baker}
%\begin{eqnarray}
%   P(L_n|\mbox{correct}) &=& P(T)+\frac{\left(1-P(T)\right) P(L_{n-1})
%          \left(1-P(S)\right)}
%               {P(L_{n-1})\left(1-P(S)\right)+\left(1- P(L_{n-1})\right) P(G)}\\
%   P(L_n|\mbox{incorrect}) &=& P(T)+\frac{\left(1-P(T)\right) P(L_{n-1})
%          \P(S)\right}
%               {P(L_{n-1})P(S)+\left(1- P(L_{n-1})\right) \left(1-P(G)\right)}\\
%with the substitution:
%
%\begin{equation}
%         P(L_{n-1}) \to P(T)+\left(1-P(t)\right) p(L_t)
%\end{equation}
%
%This equivalence is from changing the order of updating
%student mastery and updating the estimate based on student
%response, as noted by the authors.  However, this raises
%a question for Equations (3) and (4) of \cite{brunskill}.
%Should the probability be taken before or after updating the
%student mastery:



\begin{thebibliography}{9}

\bibitem{anderson} 
  Corbett, A.\ T., Anderson, J.\ R. Knowledge Tracing:  Modeling 
the Acquisition of Procedural Knowledge.  \emph{User Modeling and
 User-Adapted Interaction}, 1995, 4, 253--278.

\bibitem{qualitative}Qualitative methods for HMMs.  I have in mind
here the analog to qualitative methods for Diff. Eqs.

\bibitem{beckchang}
  Beck, J.\ E., Chang, K.-m.\ Identifiability: A Fundamental Problem of
  Student Modeling.
  \emph{Proceedings of the $11^{th}$ International Conference on User 
    Modeling}, 2007.

\bibitem{fix}Paper (Aleven?) where ambiguity is addressed by fixing 
a parameter.

\bibitem{bca} Baker, R, Corbett, A., Aleven, V.,  Improving Contextual 
    Models of Guessing and Slipping with a Truncated Training Set. 
    First International Conference on Educational Data Mining. 2008. 

\bibitem{brunskill}
   Lee, J.\ I., Brunskill, E.\ The impact of Individualizing Student 
  Models on Necessary Practice Opportunities.  EDM, 2012.

\bibitem{powerlaw}Heathcote, A., Brown, S., Mewhort, D. 
  The power law repealed: The case for an exponential law of practice.
  Psychonomic Bulletin \& Review, 2000, 7, 185-207.

\end{thebibliography}




\end{document}